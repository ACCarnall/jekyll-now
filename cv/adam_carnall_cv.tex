\documentclass[a4paper,fleqn,usenatbib,onecolumn]{mnras}

%\usepackage{newtxtext,newtxmath}

\usepackage[T1]{fontenc}
\usepackage{ae,aecompl}
\usepackage{bibentry}
\usepackage{graphicx}	% Including figure files
\usepackage{amsmath}	% Advanced maths commands
\usepackage{amssymb}	% Extra maths symbols
\usepackage{indentfirst}
\usepackage{longtable}

\title[Adam Carnall]{\center{Adam Carnall}}

\author[Adam Carnall]{Postdoctoral Research Assistant, Institute for Astronomy, Royal Observatory Edinburgh, EH9 3HJ
\\
}


\begin{document}
\maketitle

% Abstract of the paper
%\begin{abstract}

%\end{abstract}

\large

\begingroup
\setlength{\tabcolsep}{0pt} % Default value: 6pt
\renewcommand{\arraystretch}{1.1} % Default value: 1
\begin{tabular}{ p{3cm} p{14cm} }
\\
\bf{2019 - present} & \bf{Postdoctoral Research Assistant -- Royal Observatory Edinburgh}\\
& Supervisors: Prof. James Dunlop and Prof. Ross McLure\\
\\
\bf{2015 - 2019} & \bf{PhD Astrophysics -- Royal Observatory Edinburgh}\\
& Thesis: The star-formation histories of massive quiescent galaxies \\
& Supervisors: Prof. Ross McLure and Prof. James Dunlop\\
\\
\bf{2011 - 2015} & \bf{MPhys Physics and Astronomy -- Durham University} \\          
& First class honours: final mark 82\% \\
& Thesis: A new search for high-redshift quasars\\
& Supervisor: Prof. Tom Shanks\\
\end{tabular}
\endgroup

\section*{Research Interests}

\noindent Galaxy formation and evolution, quiescent galaxies, quenching mechanisms, star-formation histories, spectral fitting, spectroscopic surveys, dust attenuation, UVJ diagnostics, Python software development and distribution, Bayesian statistical methods and their computational implementation, high performance computing in astronomy


\section*{First-authored publications}
\nobibliography{adam_carnall_cv}
\bibliographystyle{plainnat}

\hangindent=0.45cm \noindent 5. \bibentry{Carnall2019b}

\medskip

\hangindent=0.45cm \noindent 4. \bibentry{Carnall2019a}

\medskip

\hangindent=0.45cm \noindent 3. \bibentry{Carnall2018}

\medskip

\hangindent=0.45cm \noindent 2. \bibentry{Carnall2017}

\medskip

\hangindent=0.45cm \noindent 1. \bibentry{Carnall2015}

\section*{Coauthored Publications}

\hangindent=0.45cm \noindent 8. \bibentry{Cullen2019}

\medskip

\hangindent=0.45cm \noindent 7. \bibentry{Kemp2019}

\medskip

\hangindent=0.45cm \noindent 5. \bibentry{Leja2019}

\medskip

\hangindent=0.45cm \noindent 6. \bibentry{Marchi2019}

\medskip

\hangindent=0.45cm \noindent 4. \noindent \bibentry{McLure2018}

\medskip

\hangindent=0.45cm \noindent 3. \noindent \bibentry{Pentericci2018}

\medskip

\hangindent=0.45cm \noindent 2. \bibentry{Chehade2018}

\medskip

\hangindent=0.45cm \noindent 1. \bibentry{Cullen2018}

\clearpage

\section*{Accepted telescope proposals}
\begingroup
\setlength{\tabcolsep}{0pt} % Default value: 6pt
\renewcommand{\arraystretch}{1.05} % Default value: 1
\begin{longtable}{ p{2cm} p{15cm} }

\bf{2019} & \bf{The stellar mass-metallicity relation for massive quiescent galaxies at 1.0 < z < 1.5}\\
& PI, 64 hours, ESO P104, VLT KMOS, 0104.B-0885\\

\\
\bf{2015} & \bf{Probing the epoch of reionisation with two bright quasars at z > 6 from VST ATLAS}\\
& Co-I (PI: T. Shanks), 2 hours, ESO P94, VLT X-SHOOTER, 294.A-5031\\

\end{longtable}
\endgroup


\section*{Awards and Prizes}
\begingroup
\setlength{\tabcolsep}{0pt} % Default value: 6pt
\renewcommand{\arraystretch}{1.4} % Default value: 1
\begin{longtable}{ p{2cm} p{15cm} }

\bf{2018} & Scottish Universities Physics Alliance PECRE Bursary: \pounds1500 travel funding\\

\bf{2015} & Durham University J. A. Chalmers Prize in Experimental Physics\\

\bf{2015} & Durham University Summer Research Placement Bursary: \pounds1500 funding for 6 week project\\

\bf{2014} & Institute of Physics Top 50 Award: \pounds2000 funding for 8 week project at Southampton University \\

\bf{2013} & Oxford University Summer Research Programme: \pounds1500 funding for 8 week summer project\\

\bf{2013} & Leicester University SURE Bursary: \pounds2000 funding for 6 week summer project\\

\end{longtable}
\endgroup


\section*{Talks and Presentations}
\begingroup
\setlength{\tabcolsep}{0pt} % Default value: 6pt
\renewcommand{\arraystretch}{1.4} % Default value: 1
\begin{longtable}{ p{2.5cm} p{2.5cm} p{12cm} }

\bf{July 2019} & Talk & Galaxy evolution session, National Astronomy Meeting, Lancaster, UK\\

\bf{July 2019} & Talk & MOONS session, National Astronomy Meeting, Lancaster, UK\\

\bf{May 2019} & Invited talk & Lega-C team meeting, Ghent, Belgium\\

\bf{Mar 2019} & Talk & Geneva Observatory, Switzerland\\

\bf{Jan 2019} & Invited talk & The growth of galaxies in the early Universe V, Sexten, Italy\\

\bf{Nov 2018} & Talk & IAU Symposium 341: Challenges in panchromatic galaxy modelling with next generation facilities, Osaka University, Japan\\

\bf{Oct 2018} & Talk & University of St Andrews, UK\\

\bf{Oct 2018} & Seminar & Royal Observatory Edinburgh, UK\\

\bf{Sep 2018} & Talk & VANDELS collaboration meeting, Arcetri Observatory, Florence, Italy\\

\bf{Apr 2018} & Invited talk & The art of measuring galaxy physical parameters, UC Riverside, USA\\

\bf{Jan 2018} & Invited talk & The growth of galaxies in the early Universe IV, Sexten, Italy\\

\bf{Jan 2018} & Talk & Durham Edinburgh Extragalactic Conference XIV, Durham, UK\\

\bf{Nov 2017} & Talk & Royal Society of Edinburgh Cormack Meeting, Edinburgh, UK\\

\bf{Sep 2017} & Talk & VANDELS collaboration meeting, Bologna Observatory, Italy\\

\bf{Jun 2017} & Invited talk & Advances in Galaxy Evolution with Surveys, Ringberg Castle, Germany\\

\bf{Jun 2016} & Poster & 32nd IAP Colloquium: Cosmic dawn of galaxy formation: linking observations and theory with new-generation spectral models, Paris, France \\

\bf{Jan 2016} & Talk & Durham Edinburgh Extragalactic Conference XII, Durham, UK\\

\end{longtable}
\endgroup


\section*{Teaching Experience}
\begingroup
\setlength{\tabcolsep}{0pt} % Default value: 6pt
\renewcommand{\arraystretch}{1.4} % Default value: 1
\begin{longtable}{ p{3cm} p{14cm} }

\bf{2018} & Edinburgh University Numerical Recipes Course: guest lecture on MCMC methods\\

\bf{2018} & St. Andrews University Research Methods Course: guest lecture on star-formation histories\\

\bf{2015 - 2019} & Edinburgh University Teaching Assistant: supervised a variety of tutorials and labs\\

\end{longtable}
\endgroup

\section*{Supervision Experience: Summer Projects}
\begingroup
\setlength{\tabcolsep}{0pt} % Default value: 6pt
\renewcommand{\arraystretch}{1.4} % Default value: 1
\begin{longtable}{ p{2cm} p{15cm} }

\bf{2019} & Edinburgh Physics Summer Scholarship, Sam Walker, Finding the first quiescent galaxies\\

\bf{2018} & Edinburgh Physics Summer Scholarship, Jamie Yellen, Advanced Bayesian methods for galaxy spectral fitting\\

\bf{2017} & Royal Society of Edinburgh Cormack Scholarship, Joe Cairns (now PhD student at Imperial College London), SCUBA-diving into the deep universe: the origins of massive galaxies\\

\end{longtable}
\endgroup


\section*{Outreach and Public Engagement}
\begingroup
\setlength{\tabcolsep}{0pt} % Default value: 6pt
\renewcommand{\arraystretch}{1.4} % Default value: 1
\begin{longtable}{ p{2cm} p{15cm} }

\bf{2018} & Royal Observatory Edinburgh Open Days talk: How many stars are there?\\

\bf{2018} & Royal Observatory Edinburgh Open Days: organised activity stall explaining spectroscopic surveys using Sloan Digital Sky Survey plate, fairy lights and portable spectrograph\\

\bf{2017} & Royal Observatory Edinburgh Open Days talk: Astronomical archaeology: how did galaxies form?\\

\bf{2015} & Teeside Skeptics talk: What are dark matter and dark energy?\\

\end{longtable}
\endgroup


\end{document}
